\section{Introduction}\label{sec:intro}

Universal Dependencies (UD) treebanks, a multilingual collection of dependency treebanks based on a shared, cross-lingually consistent annotation scheme \citep{nivre2020} and covering 138 languages with 243 treebanks in its most recent \texttt{v2.11} release \citep{universaldep}, have enabled significant advances in the development of multilingual dependency parsers and other NLP technologies \citep{zeman2017, zeman2018}. This proposed thesis will explore their potential in typology research through a cross-lingual quantitative study of verbal valency systems.

The starting point of this study is the assumption, consistent with those behind \citet{levin1993} and other work on \textit{verb classes}, that the syntactic behavior of verbs are at least in part determined by their lexical semantics, and that, as such, verb classes based on their syntactic distribution should be semantically coherent as well. This study will test this assumption computationally by performing clustering experiments on a subset of UD treebanks in order to explore whether the UD annotations support an automated induction of the valency frames in a language and whether verb classes can be further inducted based on the distribution of verbs across the valency frames. In the process of the experiments, factors that have an impact on the outcome of clustering, particularly with respect to data quantity and quality, as well as typological features of languages, will be examined. The results of these clustering experiments will then, in combination with a computationally derived cross-lingual lexicon, support typological investigations into possible universals in the organization of verbal lexicon.

This proposal itself is organized as follows: relevant literature on valency theory and dependency grammar is surveyed in \S\ref{sec:background} to provide the theoretical background for the proposed study; \S\ref{sec:rqs} formulates the key research questions the study seeks to address; \S\ref{sec:methodology} lays out the data sources and potential methodology to be used in the study; \S\ref{sec:plan} provides a work plan and suggests a tentative timeline for completing the thesis; \S\ref{sec:conclusion} concludes the proposal.