\section{Research questions}\label{sec:rqs}

The aim of this thesis study is twofold: the first is exploratory and computational, namely whether the existing computational resources based on dependency grammar can be effectively utilized in quantitative typology; the second is investigative and typological, whether a corpus-based study of valency features reveals patterns or universals in how languages organize their valency systems.

\citet{levin1993} observes in her study of English verb classes that
\begin{quote}
    Distinctions induced by diathesis alternations help to provide insights into verb meaning, and more generally into the organization of the English verb lexicon, that might not otherwise be apparent, bringing out unexpected similarities and differences between verb. (p.15)
\end{quote}

A typological study extends this aims to examine linguistic universals in verb lexicon organization that may not be obvious within a monolingual study. Will we see cross-lingual patterns or universals in how verb classes aggregate and within each cross-lingual clusters, different strategies being used to encode the verb classes? If a universal exists for a hierarchy of transitive verbs for example, this should show up in the verb classes. In construction grammar: different strategies for the same construction, e.g. adpositions and case markings

If we consider valency frames primarily a feature of the verbal lexicon, then cross-lingually, this means the comparison of the distribution of verbs across different verb classes and valency frames allows us to test possible universals regarding the organization of the verbal lexicon. The object of cross-lingual comparison therefore is crucially \textit{not} the valency frames or verb classes themselves, but the organization of the frames and classes.

Difficulty in a finite categorical classification of valency class systems, can be overcome through statistical, information theoretic methods.
