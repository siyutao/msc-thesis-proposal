\section{Research questions}\label{sec:rqs}

The aim of this thesis study is twofold: the first is exploratory and computational, namely whether the existing computational resources based on dependency grammar can be effectively utilized in quantitative typology; the second is investigative and typological, whether a corpus-based study of valency features reveals universal patterns in how languages organize their valency systems and possibly their strategies.


\citet{levin1993} Levin observes in her study of English verb classes that
\begin{quote}
    Distinctions induced by diathesis alternations help to provide insights into verb meaning, and more generally into the organization of the English verb lexicon, that might not otherwise be apparent, bringing out unexpected similarities and differences between verb. (p.15)
\end{quote}

A typological study then aims to examine these linguistic universals.

Will we see cross-lingual patterns or universals in how verb classes aggregate and within each cross-lingual clusters, different strategies being used to encode the verb classes. If a semantic universal exists for different levels of transitivity for example, this should show up in the verb classes.

Different strategies for the same construction, e.g. adpositions and case markings


The difficulty in a finite categorical classification of valency class systems can thus be overcome through statistical, information theoretic methods.
