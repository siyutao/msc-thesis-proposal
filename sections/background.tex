\section{Background and Motivation}

\subsection{Valency and valency phenomena}


Linguistics borrowed the term of \textbf{valency} from chemistry, as the capacity of a verb to combine with its arguments are reminiscent of the combining capacity of an atom of a given element.

The Lucien Tesnière is generally credited as introducing the term of valency in his posthumously published \textit{Éléments de syntaxe structurale} (\cite*{tesniere1959}; English translation \cite*{tesniere2015}). But the study of the phenomena we term valency doubtlessly predates that. \footnote{
    \citet{przepiorkowski2018} notes that so does the analogy with chemistry, which made its first appearance as early as in \citet{peirce1897}.
}

The issue of argument encoding lies at the center of studies into sentence meaning. Most linguistic theories, as well as psycholinguistic evidence, point to the centrality of the verb in determining both the structure and meaning of a sentence. Beyond that, however, approaches to the study of argument encoding differ.

In generative grammar, syntactic valency of a verb is studied in terms of ``subcategorization frames'' encoded in the lexicon. \todo{describe Chomskyian approach re subcategorization}

\citet{levin1993}'s seminal work provides much more fine-grained categorization of verbs based on their syntactic behavior. \todo{levin}

\citep{fillmore1967,fillmore1970} 

\citet{levin2005} identifies five major questions that are necessary for a compelte theory of argument realization.
\todo{copy from hand notes}

Construction grammar approach:
CxG would consider valency frame as a level of construction. Whether or not this construction is autonomous will depend on whether the unpredictability condition is satified - in so far that the properties of valency frame cannot be predicted from other grammatical units.

The relevance of a typological study 

\citet{croft2012} takes a CxG approach. 


\subsection{Contrastive studies of valency}


Typological interest in valency is primarily focused on cross-linguistic mismatches, termed \textit{metataxis} by \citet{tesniere1959}.

One of the key question is also whether this valency frame are syntactically defined or semantically so. Since in the study of argument structure, we're dealing with the syntactic expression of lexical and non-lexical semantics, this question is particularly murky to answer.

\citet{tsunoda1981, tsunoda1985} proposes an hierarchy of verbs



\subsection{Dependency grammars}

Universal Dependencies 



