\section{Background and Motivation}

\subsection{Argument Structure: Generative and Construction Grammar Approaches}


Generative approach:

Levin (1993)'s verb classes 

Levin and Rappaport Hovav (2005) identifies five major questions that are necessary for a compelte theory of argument realization.
\todo{copy from hand notes}

Croft (2012) inter alia

Construction grammar approach:
CxG would consider valency frame as a level of construction. Whether or not this construction is autonomous will depend on whether the unpredictability condition is satified - in so far that the properties of valency frame cannot be predicted from other grammatical units.

The relevance of a typological study 

\subsection{Valency and Verb Classes}

Linguistics borrowed the term of \textbf{valency} from chemistry, as the capacity of a verb to combine with its arguments are reminiscent of the combining capacity of an atom of a given element.

Typological interest in valency is primarily focused on cross-linguistic mismatches, termed \textit{metataxis} by \citet{tesniere1959}.

One of the key question is also whether this valency frame are syntactically defined or semantically so. Since in the study of argument structure, we're dealing with the syntactic expression of lexical and non-lexical semantics, this question is particularly murky to answer.

\citet{tsunoda1981, tsunoda1985} proposes an hierarchy of verbs



\subsection{Dependency Grammars}
In fact, modern developments in valency theory and dependency grammar can both trace their roots to the work of the French linguist Lucien Tesnière, posthumously published in \textit{Éléments de syntaxe structurale} (\cite*{tesniere1959}; English translation \cite*{tesniere2015}).

Universal Dependencies 



