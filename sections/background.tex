\section{Background and Motivation}

\subsection{Valency and valency phenomena}

% valency - terminology introduction

In chemistry, \textit{valency}, or \textit{valence}, refers to the combining power of an atom or radical. The valency of any atom can be measured by the number of hydrogen atoms that it can combine with or displace in a chemical compound \citep{law2020a}. This same term has been used in linguistics to similar effect and refers to the combining power of a word, primarily a verb or other predicate, with other words or elements of the sentence. 

Lucien Tesnière is generally credited with introducing the term valency to linguistics with his syntactic theory of valency and dependence, as presented in the posthumuously published \textit{Éléments de syntaxe structurale} (\cite*{tesniere1959}; English translation \cite*{tesniere2015}).\footnote{
    As \citet{przepiorkowski2018} notes, while Tesnière is rightly credited with the introduction of a theory of linguistic valency, the metaphor of valency itself has made appearances as early as in \citet{peirce1897}, among others.
}
In another of Tesnière's metaphors, each verbal node, being the center of sentence structure, is not unlike a ``theatrical performance'' with the verb expressing the process and the nouns being the \textit{actants} (what we would now call \textit{arguments}) in this performance. Just like atoms of different elements allow for a greater or lesser number of bonds, different verbs can combine with a greater or lesser number of actants, i.e., their valency.

% the phenomena now we are now calling valency - different theories
While the term valency is borrowed into linguistics from chemistry, the study of phenomena that are covered by or overlap with valency has a much longer tradition, dating to the early beginnings of linguistics from the kāraka level of semantic relation between verb and noun \citep{ganeri2011a} in Pāṇinian grammar to modern case grammar \citep{fillmore1967}. 

Most linguistic theories, as well as psycholinguistic evidence, point to the centrality of the verb in determining both the structure and meaning of a sentence. This places valency and the issues of \textit{argument encoding} squarely at the center.

Beyond that, however, approaches to the study of argument encoding differ.

%% generative grammar - subcategorization
In generative grammar, syntactic valency of a verb is studied in terms of ``subcategorization frames'' encoded in the lexicon. \todo{describe Chomskyian approach re subcategorization}

\citet{levin1993}'s seminal work provides much more fine-grained categorization of verbs based on their syntactic behavior. \todo{levin}

%% frame semantics - fillmore 
\citep{fillmore1967,fillmore1970} 


%% construction grammar
CxG would consider valency frame as a level of construction. Whether or not this construction is autonomous will depend on whether the unpredictability condition is satified - in so far that the properties of valency frame cannot be predicted from other grammatical units.

\citet{croft2012} takes a CxG approach. 


\subsection{Cross-lingual and contrastive studies of valency}

% why approach valency from typological perspective
Typological interest in valency is primarily focused on cross-linguistic mismatches.

Already \citet{tesniere1959} describes the process of \textit{metataxis}, by which syntactic structures of one language is translated to those of another.  

\citet{levin2005} identifies five major questions that are necessary for a compelte theory of argument realization.
\todo{copy from hand notes and add how a typological study can help answer some of these questions}

% monolingual valency dictionaries

One of the key question is also whether this valency frame are syntactically defined or semantically so. Since in the study of argument structure, we're dealing with the syntactic expression of lexical and non-lexical semantics, this question is particularly difficult to answer.

% cross-lingual contrastive studies

% typological approaches
\citet{tsunoda1981, tsunoda1985} proposes an hierarchy of verbs



\subsection{Dependency grammars}

Universal Dependencies 



