\section{Conclusion}\label{sec:conclusion}

The proposed thesis would like to contribute to the study of verbal valency systems by including the typological and quantitative perspectives. It will attempt novel combinations of computational methods and corpora in service of a quantitative typological investigation, therefore exploring the utility and limits of dependency-based datasets and automatic clustering methods. In doing so, it also aims to contribute to the typological study and theoretical debate on verbal valency. It is hoped that the difficulty in working out a finite categorical typology of valency class systems may be overcome through statistical and information-theoretic methods and modern computational corpora; and that furthermore the empirical study can bring out patterns and observations that may not have been obvious in an introspective study and help settle or intensify theoretical debates on valency.

% \citet{levin1993} observed in her study of English verb classes that
% \begin{quote}
%     Distinctions induced by diathesis alternations help to provide insights into verb meaning, and more generally into the organization of the English verb lexicon, that might not otherwise be apparent, bringing out unexpected similarities and differences between verb. (p.15)
% \end{quote}
